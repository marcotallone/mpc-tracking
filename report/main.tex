\documentclass{settings/notex}

% PREAMBLE ─────────────────────────────────────────────────────────────────────

% Images path
\graphicspath{{images/}}

% Additional packages here
\usepackage{blindtext}
\usepackage{siunitx}
\usepackage{makecell}
\RequirePackage{emoji}
\setemojifont{Noto Color Emoji}
\RequirePackage{fontawesome5}

% Subfiles package (best loaded last in the preamble)
\usepackage{subfiles}

% TITLE, AUTHOR AND DATE ───────────────────────────────────────────────────────

% Title, Author and Date
\title{Non-linear Reference Tracking via\\
	Model Predictive Control and Extended Kalman Filter\\
  \vspace{0.5cm}
  \fontsize{12pt}{12pt}\selectfont{
    Modelling and Control of Cyber-Physical Systems II\\
    \vspace{0.25cm}
    University of Trieste (UniTS)
  }
}

\author{Marco Tallone}
\date{January 2025}

% DOCUMENT ─────────────────────────────────────────────────────────────────────

\begin{document}

\maketitle

\begin{abstract}
\noindent
This report presents the implementation of a Model Predictive Control (MPC)
algorithm and an Extended Kalman Filter (EKF) to perform reference tracking of
non-linear dynamical systems. The implemented algorithms have been tested and assessed on different non-linear systems and multiple
trajectories. Results confirm the effectiveness of the proposed MPC approach in
tracking the desired trajectories, even in the presence of noise and
disturbances through the use of the EKF.
\end{abstract}

% \tableofcontents

% Sections
\subfile{sections/01-introduction}
\subfile{sections/02-models}
\subfile{sections/03-trajectory-generation}
\subfile{sections/04-mpc}


% Bibliography
% Remember to compile with the sequence:
% lualatex -> bibtex -> lualatex -> lualatex
\pagebreak
\bibliographystyle{plainurl}
\bibliography{bibliography}

% % Appendix
% \pagebreak
% \appendix
% \renewcommand{\thesection}{\Alph{section}} % Section numbering to A, B, C, ...
% \renewcommand{\thesubsection}{\thesection\arabic{subsection}} % Subsection numbering to A1, A2, ...
% \subfile{sections/appendix}

\end{document}
