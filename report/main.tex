\documentclass{settings/notex}

% PREAMBLE ─────────────────────────────────────────────────────────────────────

% Images path
\graphicspath{{images/}}

% Additional packages here
\usepackage{blindtext}
\usepackage{siunitx}
\usepackage{makecell}
\RequirePackage{emoji}
\setemojifont{Noto Color Emoji}
\RequirePackage{fontawesome5}

% Subfiles package (best loaded last in the preamble)
\usepackage{subfiles}

% TITLE, AUTHOR AND DATE ───────────────────────────────────────────────────────

% Title, Author and Date
\title{Non-linear Reference Tracking via\\
	Model Predictive Control and Extended Kalman Filter\\
  \vspace{0.5cm}
  \fontsize{12pt}{12pt}\selectfont{
    Modelling and Control of Cyber-Physical Systems II\\
    \vspace{0.25cm}
    University of Trieste (UniTS)
  }
}

\author{Marco Tallone}
\date{January 2025}

% DOCUMENT ─────────────────────────────────────────────────────────────────────

\begin{document}

\maketitle

\begin{abstract}
\noindent
This report presents the implementation of a Model Predictive Control (MPC)
algorithm and an Extended Kalman Filter (EKF) to perform reference tracking of
non-linear dynamical systems. The implemented algorithms have been tested and assessed on different non-linear systems and multiple
trajectories. Results confirm the effectiveness of the proposed MPC approach in
tracking the desired trajectories, even in the presence of noise and
disturbances through the use of the EKF.
\end{abstract}

% \tableofcontents

% Sections
\subfile{sections/01-introduction}
\subfile{sections/02-models}
\subfile{sections/03-trajectory-generation}
\subfile{sections/04-mpc}
\subfile{sections/05-ekf}
\subfile{sections/06-results}
\pagebreak
\subfile{sections/07-conclusions}

% Bibliography
% Remember to compile with the sequence:
% lualatex -> bibtex -> lualatex -> lualatex
\pagebreak
\bibliographystyle{plainurl}
\bibliography{bibliography}

\vspace{1cm}

\section*{Generative Tools Notice}

Generative AI tools have been used as a support for the development of some
parts of this
project. In particular, the
\href{https://en.wikipedia.org/wiki/Microsoft_Copilot}{Copilot \faLink} generative tool
based on \href{https://en.wikipedia.org/wiki/GPT-4}{OpenAI GPT 4o \faLink} model has
been used as assistance medium in performing the following tasks:

\begin{itemize}
	 \item writing documentation and comments in the implemented models for
		 \ttt{MATLAB}'s \ttt{help} function

	\item fixing implementation bugs in the \ttt{MPC.dense\_formulation()}
		method used to define the QP problem matrices in dense MPC formulation

	 \item improving variable naming and overall code readability

	\item grammar and spelling check both in the repository
		\ttt{README}~\cite{github} and in this report

	\item aesthetic improvements in the plots of this report

\end{itemize}

Nevertheless, the author assumes full responsibility for the final content and
correctness of the project.

% % Appendix
\pagebreak
\appendix
\renewcommand{\thesection}{\Alph{section}} % Section numbering to A, B, C, ...
\renewcommand{\thesubsection}{\thesection\arabic{subsection}} % Subsection numbering to A1, A2, ...
\subfile{sections/appendix}

\end{document}
