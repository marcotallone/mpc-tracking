\documentclass[../main.tex]{subfiles}
\graphicspath{{\subfix{../images/}}} % Images path

\begin{document}

\section{Non-linear Models}\label{sec:models}

In order to test and asses the performance of the implemented MPC and EKF
algorithms, two systems of different complexity have been considered. In particular, these are:

\begin{itemize}
	\item A \bft{Unicycle model}
	\item A \bft{Helicopter model}
\end{itemize}

Both systems are mathematically described by non-linear dynamics.

\subsection{Unicycle Model}

The first system is a relatively simple unicycle model described by the
non-linear dynamics in equations~\eqref{eq:unicycle-dynamics}.
This consists of a rigid body moving on a plane with two parallel wheels of radius
$r$ separated by a distance $L$.
\begin{equation}\label{eq:unicycle-dynamics}
  \begin{cases}
	  \dot{x} = \frac{r}{2} (\omega_1 + \omega_2) \cos(\theta)\\
	  \dot{y} = \frac{r}{2} (\omega_1 + \omega_2) \sin(\theta)\\
	  \dot{\theta} = \frac{r}{L} (\omega_1 - \omega_2)
  \end{cases}
\end{equation}

The state of the system is hence defined by the Cartesian coordinates $x$ and
$y$ of the center of mass of the unicycle and the heading angle $\theta$. The
inputs are instead the angular velocities of the two wheels $\omega_1$ and
$\omega_2$.
\begin{equation*}
	\mbf{x}(t) = \begin{bmatrix} x(t) & y(t) & \theta(t) \end{bmatrix}^T
	\quad \text{and} \quad
	\mbf{u}(t) = \begin{bmatrix} \omega_1(t) & \omega_2(t) \end{bmatrix}^T
\end{equation*}

\subsection{Helicopter Model}

The second system represents instead a miniature helicopter model taken from the
paper by \itt{Kunz, Huck and Summers}~\cite{helicopter}. As explained by the
authors, the dynamics can be described by the Newton-Euler laws of mechanics for
single rigid bodies. However, a simplification has been made in favor of an
easier implementation of the MPC algorithm for this model. In particular,
introducing the assumption that the pitch and roll inputs directly act on the
translational accelerations, it's possible to omit the pitch and roll angular
states of the helicopter body. The result is the non-linear dynamics in
equations~\eqref{eq:helicopter-dynamics}.
\begin{equation}\label{eq:helicopter-dynamics}
	\begin{cases}
		\dot{x}_I = cos(\psi) \dot{x}_B - sin(\psi) \dot{y}_B\\
		\dot{y}_I = sin(\psi) \dot{x}_B + cos(\psi) \dot{y}_B\\
		\dot{z}_I = \dot{z}_B\\
		\ddot{x}_B = b_x u_x + k_x \dot{x}_B + \dot{\psi} \dot{y}_B\\
		\ddot{y}_B = b_y u_y + k_y \dot{y}_B - \dot{\psi} \dot{x}_B\\
		\dot{z}_B = b_z u_z - g\\
		\dot{\psi} = \dot{\psi}\\
		\ddot{\psi} = b_{\psi} u_{\psi} + k_{\psi} \dot{\psi}
	\end{cases}
\end{equation}

where $x_I$, $y_I$, $z_I$ denote the position of the helicopter in the
inertial frame, $\dot{x}_B$, $\dot{y}_B$, $\dot{z}_B$ the velocities of the
helicopter in the body frame, $\psi$ the yaw heading angle and $\dot{\psi}$ its
rotational velocity in yaw. The control inputs for pitch, roll, thrust and yaw
are instead denoted by $u_x$, $u_y$, $u_z$, $u_{\psi}$.\\
Due to implementation necessities, the only difference with respect to the
original paper is the absence of the augmented integral states which were used
to reduce steady-state error. The rest of the setup is consistent with the
original work.\\
In this case we hence obtained a model having $8$ state components and $4$
control inputs formalized as follows.

\begin{equation*}
	\begin{aligned}
		\mbf{x}(t) &= \begin{bmatrix} x_I(t) & y_I(t) & z_I(t) & \dot{x}_B(t) &
		\dot{y}_B(t) & \dot{z}_B(t) & \psi(t) & \dot{\psi}(t) \end{bmatrix}^T\\
			&\text{and}\\
		\mbf{u}(t) &= \begin{bmatrix} u_x(t) & u_y(t) & u_z(t) & u_{\psi}(t) \end{bmatrix}^T
	\end{aligned}
\end{equation*}

\end{document}

