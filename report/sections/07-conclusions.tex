\documentclass[../main.tex]{subfiles}
\graphicspath{{\subfix{../images/}}} % Images path

\begin{document}

\section{Conclusions}\label{sec:conclusions}

In conclusion, this study has shown the implementation of a Model Predictive
Control (MPC) algorithm and an Extended Kalman Filter (EKF) to perform reference
tracking of non-linear dynamical systems in a fast and efficient way.
The MPC algorithm developed has been successfully tested on both the unicycle
and the helicopter models, showing the ability to track different reference
trajectories with high precision, even in the presence of noise and
disturbances thanks to the addition of the EKF.\\
From the simulations conducted on the unicycle model, it has been shown that the
MPC controller is able to track a reference trajectory with good precision
and the variability of the average RMSE is minimal. In both trajectories the state RMSE is of the order of $10^{-2}$, which comes at
the cost of a slightly higher (but still acceptable) error on the control input of the order of $10^{-1}$.\\
Extremely satisfying results
have also been obtained from the helicopter model on the circular trajectory,
with average RMSE of $\sim 0.03$ and $\sim 0.09$ for state and input
respectively that confirm in this way the findings by \itt{Kunz, Huck and
Summers}~\cite{helicopter}. The helicopter model performed significantly worse
on the lemniscate trajectory, which is intrinsically more challenging due to the 
higher number of turns in different directions. Here the errors on the state and
input are both of the order of $10^{-1}$, meaning that either the penalization
applied is not sufficient to guarantee correct tracking or that the reference
input computed for the original non-linear model is too far from the one needed by
the approximate LTV model internally used in the MPC controller. Further test
with different penalization matrices might have to be conducted to address this
issue.\\
Expectedly, the results obtained from the simulations with the presence of noise
show a significant increase in the RMSE of the state and input, with 
higher variability given the increased values of standard deviation. Still,
recalling that in this case the algorithm uses the EKF 
only in the first step of the MPC prediction horizon (without propagating any
uncertainty further), the results are satisfactory and not only show the
effectiveness of the EKF in estimating the states of the system, but also the
robustness of the MPC controller under these conditions.\\
From figure~\ref{fig:horizon} it's also possible to observe the effect of the
MPC horizon length $N$ on the final measured errors. For the unicycle model, an
increase in the horizon length leads to a monotone decrease in the state RMSE,
while the corresponding input RMSE saturates to a constant value.
This behavior must probably be due to the penalization chosen,
which clearly weights state errors more than input ones. For the
helicopter model instead, the measured errors oscillates for low values of the
prediction horizon, but in this case the result is most probably dependent on
the particular initial condition chosen for the simulation. However, for higher
values of the horizon length, ranging from $N=6$ to $N \approx 15$, the errors
seem to stabilize around a certain value, which is probably the best trade-off
between the computational load and the tracking performance. Additionally, this
suggests that for the lemniscate trajectory and the given setup a shorter horizon
than the one proposed by the original authors ($N=18$) might be more suitable.
Finally, notice how both state and input RMSE start to increase for $N > 15$. A
possible explanation for this behavior might be the fact that the MPC controller
is trying to react too quickly to the future reference points, leading to a
worse tracking of the closer~ones.

\end{document}

