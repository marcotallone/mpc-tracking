\documentclass[../main.tex]{subfiles}
\graphicspath{{\subfix{../images/}}} % Images path

\begin{document}

\section{Conclusions}\label{sec:conclusions}

In conclusion, this study has shown the implementation of a Model Predictive
Control (MPC) algorithm and an Extended Kalman Filter (EKF) to perform reference
tracking of non-linear dynamical systems in a fast and efficient way.
The MPC algorithm developed has been successfully tested on both the unicycle
and the helicopter models, showing the ability to track different reference
trajectories with high precision, even in the presence of noise and
disturbances thanks to the addition of the EKF.\\
From the simulations conducted on the unicycle model, it has been shown that the
MPC controller is able to track the reference trajectory with good precision and
that the results do not depend on the specific initial condition given the
minimal variability. In both trajectories the state RMSE is of the order of $10^{-2}$, which comes at
the cost of a slightly higher (but still acceptable) error on the control input
with magnitude $\sim 10^{-1}$.\\
Extremely satisfying results
have also been obtained from the helicopter model on the circular trajectory,
with average RMSE of $\sim 0.03$ and $\sim 0.09$ for state and input
respectively that confirm in this way the findings by \itt{Kunz, Huck and
Summers}~\cite{helicopter}. The helicopter model performed noticeably worse
on the lemniscate trajectory, which is intrinsically more challenging due to the 
higher number of turns in different directions. Here the errors on the state and
input are both of the order of $10^{-1}$, meaning that either the penalization
applied is not sufficient to guarantee correct tracking or that the reference
input computed for the original non-linear model is too far from the one needed by
the approximate LTV model internally used in the MPC controller. Further test
with different penalization matrices might have to be conducted to address this
issue.\\
Expectedly, the results obtained from the simulations with the presence of noise
show a significant increase in the RMSE of the state and input, with 
higher variability given the increased values of standard deviation. Still,
recalling that in this case the algorithm uses the EKF 
only in the first step of the MPC prediction horizon (without propagating any
uncertainty further), the results are satisfactory and not only show the
effectiveness of the EKF in estimating the states of the system, but also the
robustness of the MPC controller under these conditions.\\
From figure~\ref{fig:horizon} it's also possible to observe the effect of the
MPC horizon length $N$ on the final measured errors. For the unicycle model, an
increase in the horizon length leads to an almost monotone decrease in the state RMSE,
while the corresponding input RMSE saturates to a seemingly constant value.
A similar behavior is observed for the helicopter model, but in this case it's
possible to notice that for horizon lengths $N > 9$, the state error stabilizes
(with small oscillations) around $\sim 0.03$ in agreement with the previous tests.
This suggests the possibility of using a shorter horizon than the one proposed
by the original authors ($N=18$) in order to reduce the computational load while
maintaining the same tracking performance.\\
The increase in the input RMSE recorded in both cases can be explained by the
fact that the reference inputs, computed for the original non-linear model,
might not be the most ideal ones for the approximate LTV model adopted by the
MPC controller. Hence, since the chosen penalization favors the tracking of the state over the input, the controller prioritizes the minimization of the former at the expense of the latter as it's possible to see from the results.

\end{document}

