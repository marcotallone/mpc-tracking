\documentclass[../main.tex]{subfiles}
\graphicspath{{\subfix{../images/}}} % Images path

\begin{document}

\section{Introduction}

Model Predictive Control (MPC) is a control strategy that has been widely
adopted for the control of dynamical systems. The MPC approach is based on the
solution of an optimization problem over a finite horizon, which allows for the
consideration of constraints on the system states and inputs. When used for
reference tracking, the MPC algorithm
computes the optimal control input by minimizing a cost function that penalizes
the deviation of the system states from the desired trajectory.\\
However, real-world systems are often affected by noise and disturbances, which
can lead to undesired results with the adoption of an MPC controller. Moreover,
whenever the complete state of a system cannot be fully measured, the use of
state estimators is required to infer the unmeasured components and
apply the MPC algorithm.
To address
this issues, the Extended Kalman Filter (EKF) can be used to estimate the states
of a non-linear system by fusing the available measurements with the system
dynamics and incorporating information about the process and measurement
noise.\\
This report presents the implementation of a Model Predictive Control algorithm
for non-linear dynamical systems based on successive linearization and
discretization of the dynamics around operating points, as well as the
development of an Extended Kalman Filter to estimate the states of the system in
the presence of noise.\\
The objectives of this project are both to study the effectiveness of the
non-linear MPC algorithm applied for tracking reference trajectories under
different circumstances and also to reproduce and compare the obtained results
with the ones presented by \itt{Kunz, Huck and Summers}~\cite{helicopter} in
their work on the tracking of a helicopter model.\\
The following report is structured as follows. Section~\ref{sec:models}
introduces the non-linear models used for the simulations with their dynamics,
while Section~\ref{sec:trajectory-generation} explains how feasible
trajectories can be generated for such models. Section~\ref{sec:mpc} then
describes the linear time varying (LTV) approximation of the models 
and how such approximation is used in the MPC algorithm. In
Section~\ref{sec:ekf} the addition of the Extended Kalman Filter is instead
presented.
Finally,
Section~\ref{sec:results} presents the results of the simulations followed by
some final consideration in the last section.\\
From a technical point of view all the algorithms and the models for the
dynamical systems have been developed in \ttt{MATLAB}, version
\ttt{R2024b}~\cite{matlab}. The code has been structured in a modular way to
allow for easy testing and comparison of different algorithms and models.
However, with the aim of maintaining a clear and concise report of the work, 
the following sections will mostly present the results of this study and further concepts from a theoretical 
point of view,
while implementation details and practical usage instructions will be available on the author's GitHub repository~\cite{github}.

\end{document}
