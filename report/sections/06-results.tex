\documentclass[../main.tex]{subfiles}
\graphicspath{{\subfix{../images/}}} % Images path

\begin{document}

\section{Validation and Results}\label{sec:results}

In order to validate the implemented models and the MPC algorithm, several tests
have been conducted on non-linear systems previously introduced. All the simulations
consisted in tracking a reference trajectory with the MPC controller and
measuring the average error between the reference and the actual state of the
system.

\subsection{Experimental Parameters}

Fixed values for the models parameters have been set in order to conduct the
experiments in a controlled environment. For the sake of this project, the
values here reported are expressed as dimensionless quantities since they do not
necessarily come from real measurements and they are used only to validate the
implementation from a computational point of view. However, realistic measurements can eventually be obtained from real-world systems and substituted in the models.\\
For the Unicycle model, a radius of $r = 0.03$ and a distance
between the wheels of $L = 0.3$ have been chosen. The Cartesian coordinates of
the center of mass have then been constrained in the region $[-2, 2] \times [-2,
2]$, while the input angular velocity of the wheels in the range $[-50, 50]$.\\
For the Helicopter model, the same values for the parameters reported in the
original paper~\cite{helicopter} have been used. The same state constrained have
also been maintained, but the inputs range has been amplified to $[-2, 2]$ to
account for the absence of the integral states in the model.\\
For both models, the same sampling time of $T_s = 0.1$ has been used, and no
limit has been imposed on the heading and yaw angles. Concerning the MPC
parameters instead, the following prediction horizon lengths and weighting
states and inputs matrices have been used for the two models:
\[
\begin{minipage}{.2\textwidth}
\end{minipage}%
\begin{minipage}{.3\textwidth}
  % \centering
  $\begin{array}{r@{{}\mathrel{=}{}}l}
	  N & 10 \\[\jot]
	  Q & 10^3 \cdot \mbb{I}_{3 \times 3} \\[\jot]
	  R & \mbb{I}_{2 \times 2}
  \end{array}$\\
	\vspace{0.25cm}
  \text{for the unicycle model}
\end{minipage}%
\begin{minipage}{.10\textwidth}
\quad
\end{minipage}%
\begin{minipage}{.3\textwidth}
  % \centering
  $\begin{array}{r@{{}\mathrel{=}{}}l}
	N & 18 \\[\jot] 
	Q & \text{diag}\left(50, 50, 5, 10, 3, 3, 1, 2\right) \\[\jot]
	R & 2 \cdot \mbb{I}_{4 \times 4}
  \end{array}$\\
  \vspace{0.25cm}
  \text{for the helicopter model}
\end{minipage}%
\]

The two systems have been tested on both the circular trajectory of radius $r_0
= 0.5$ and the leminscate trajectory with $a = 1$ by sampling $N_{guide} = 100$
points at regular intervals from each of them.

\subsection{Experimental Setup}

In order to perform controlled and reproducible experiments, the same setup has
been used for all the simulations. In particular, each test consisted in a full
revolution around each trajectory by starting from a random initial condition
chosen within a small neighborhood of the first reference point. Specifically,
the neighborhood has been defined as the set of points in the state space whose
Euclidean norm is less than $0.05$ from the first reference point. In order to
obtain statistically significant results, each test has been repeated $100$
times for each trajectory and each model.\\
Throughout each simulation, the Mean Squared Errors (MSE) between both the
reference state and input 
and the corresponding values assumed by the system have been computed at each time step and then averaged over
the entire trajectory. The quality of the tracking for the given simulation has
then been assessed using the final Root Mean Squared Error (RMSE) as a
performance metric. In the end, the average RMSE over the $100$ repetitions has
been computed for each trajectory and each model.\\
Simulations with the presence of noise have also been conducted in order to
compare the capabilities of the MPC controller in tracking the reference
trajectory in the presence of disturbances. In this case, only $20$ repetitions
have been conducted due to the higher computational load and the following process
noise and measurement noise matrices have been used for all the simulations:
\begin{equation*}
	\tilde{\mbf{Q}} = 0.75 \cdot 10^{-3} \cdot \mbb{I}_{n \times n},
	\quad \text{and} \quad
	\tilde{\mbf{R}} = 10^{-2} \cdot \mbb{I}_{m \times m}
\end{equation*}

where $n$ and $m$ are the dimensions of the state and input vectors of the two
models, respectively. For all the tests involving the EKF, the initial state
covariance matrix has always been initialized as $\mbf{P}_0 = \mbb{I}_{n \times
n}$.

\pagebreak
\subsection{Results}
% Unicycle Circle
% State RMSE: 0.0199 +/- 0.0072
% Input RMSE: 0.2265 +/- 0.0675

% Unicycle Leminscate
% State RMSE: 0.0298 +/- 0.0020
% Input RMSE: 0.8573 +/- 0.0230

% Helicopter Circle
% State RMSE: 0.0335 +/- 0.0126
% Input RMSE: 0.0962 +/- 0.0113

% Helicopter Leminscate
% State RMSE: 0.6863 +/- 0.0012
% Input RMSE: 0.3019 +/- 0.0071

The simulations conducted with the unicycle on the circular trajectory resulted
in an average RMSE of $0.020 \pm 0.007$ on the state of the model, while the
average input RMSE was $0.226 \pm 0.067$. On the leminscate trajectory, slightly
worse (but still satisfactory) results were obtained by recording a state RMSE
of $0.030 \pm 0.002$ and an input RMSE of $0.857 \pm 0.023$.\\
The tests conducted with the helicopter model resulted instead in an average
RMSE of $0.034 \pm 0.013$ and an input RMSE of \dots


\end{document}

