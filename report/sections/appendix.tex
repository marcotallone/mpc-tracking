\documentclass[../main.tex]{subfiles}
\graphicspath{{\subfix{../images/}}} % Images path

\begin{document}

\section{Appendix}\label{appendix}

\subsection{Murray Trajectory Generation Method}\label{app:murray}

Consider the non-linear system described by the state equation
$\mbf{x}(t) = \mbf{f}(\mbf{x}(t), \mbf{u}(t))$ and having a defined set of flat outputs
identified by $\mbf{z}(t)$.
Given a set of $N_{\text{start}}$ reference values for the flat outputs and their derivatives up to
order $q$
\begin{equation*}
	\{ \mbf{z}(T_1), \dot{\mbf{z}}(T_1), \ldots, \mbf{z}^{(q)}(T_1), \dots,
		\mbf{z}(T_{N_{\text{start}}}), \dot{\mbf{z}}(T_{N_{\text{start}}}),
	\ldots, \mbf{z}^{(q)}(T_{N_{\text{start}}}) \},
\end{equation*}

it's possible to generate a feasible trajectory for the system consisting in a
possibly larger number $N > N_{\text{start}}$ of reference values by first building
feasible parametrized trajectories between each pair of consecutive reference values and then
sampling at the desired rate.\\
To explain this method, first introduced by \itt{Murray}~\cite{murray},
consider for simplicity the first two reference values (the process is similarly
repeated for subsequent pairs). Without loss of generality, assume that
\begin{equation*}
	T_1 = 0 \quad \text{and} \quad T_2 = T
\end{equation*}

Hence, the vector $\bar{\mbf{z}}$ of known values for the flat outputs and their derivatives at
the endpoints of the considered time interval assumes the form
\begin{equation*}
	\bar{\mbf{z}} = \begin{bmatrix}
		\mbf{z}(0) & \dot{\mbf{z}}(0) & \ldots & \mbf{z}^{(q)}(0) & \mbf{z}(T) &
		\dot{\mbf{z}}(T) & \ldots & \mbf{z}^{(q)}(T)
	\end{bmatrix}^T
\end{equation*}

Since $\mbf{z}(t)$ must satisfy these boundary conditions, an analytical
expression for the flat outputs can be found by parametrizing them as a linear
combination of a set of $N_{\text{basis}}$ smooth basis functions
$\mbf{b}_i(t)$:
\begin{equation*}
	\mbf{z}(t) = \sum_{i=1}^{N_{\text{basis}}} \alpha_i \mbf{b}_i(t)
\end{equation*}

where the coefficients $\alpha_i$ are determined by the linear system of
equations obtained by imposing the boundary conditions on the basis functions
and the corresponding derivatives up to order $q$:
\begin{equation*}
	\begin{bmatrix}
		\mbf{b}_1(0) & \mbf{b}_2(0) & \ldots & \mbf{b}_{N_{\text{basis}}}(0) \\
		\dot{\mbf{b}}_1(0) & \dot{\mbf{b}}_2(0) & \ldots & \dot{\mbf{b}}_{N_{\text{basis}}}(0) \\
		\vdots & \vdots & \ddots & \vdots \\
		\mbf{b}_1^{(q)}(0) & \mbf{b}_2^{(q)}(0) & \ldots & \mbf{b}_{N_{\text{basis}}}^{(q)}(0) \\
		\mbf{b}_1(T) & \mbf{b}_2(T) & \ldots & \mbf{b}_{N_{\text{basis}}}(T) \\
		\dot{\mbf{b}}_1(T) & \dot{\mbf{b}}_2(T) & \ldots & \dot{\mbf{b}}_{N_{\text{basis}}}(T) \\
		\vdots & \vdots & \ddots & \vdots \\
		\mbf{b}_1^{(q)}(T) & \mbf{b}_2^{(q)}(T) & \ldots & \mbf{b}_{N_{\text{basis}}}^{(q)}(T)
	\end{bmatrix}
	\begin{bmatrix}
		\alpha_1 \\ \alpha_2 \\ \vdots \\ \alpha_{N_{\text{basis}}}
	\end{bmatrix}
	=
	\begin{bmatrix}
		\mbf{z}(0) \\ \dot{\mbf{z}}(0) \\ \vdots \\ \mbf{z}^{(q)}(0) \\
		\mbf{z}(T) \\ \dot{\mbf{z}}(T) \\ \vdots \\ \mbf{z}^{(q)}(T)
	\end{bmatrix}
\end{equation*}

As explained by \itt{Murray}~\cite{murray}, we obtain a linear system of the
form $M\boldsymbol{\alpha} = \bar{\mbf{z}}$. Hence, assuming that $M$ is full column rank, we can
obtain a (possibly non-unique) solution for $\boldsymbol{\alpha}$ that satisfies the
trajectory generation requirements.
In other words, the result is an analytical expression for the flat outputs
$\mbf{z}(t)$ (and their derivatives up to order $q$) between the given
endpoints.\\
This process can then be repeated for each pair of consecutive reference values
in the initial set to obtain multiple smooth point-to-point expression for the
flat outputs. Finally, the obtained trajectories can be resampled at the desired
time intervals and the corresponding values of the flat outputs can be used to
generate feasible trajectories for the system taking advantage of the
\itt{differential flatness} property.

\pagebreak
\subsection{Dense Formulation}\label{app:dense-formulation}

Given an MPC prediction horizon $N$, the \itt{dense} formulation can be obtained by first
introducing the vectorized notation
\begin{equation*}
	\begin{aligned}
		\mbf{X}_k &= \begin{bmatrix}
			 \mbf{x}_k^T &  \mbf{x}_{k+1}^T & \ldots & 
			\mbf{x}_{k+N-1}^T
		\end{bmatrix}^T\\
		\mbf{U}_k &= \begin{bmatrix}
			 \mbf{u}_k^T &  \mbf{u}_{k+1}^T & \ldots & 
			\mbf{u}_{k+N-1}^T
		\end{bmatrix}^T
	\end{aligned}
\end{equation*}

so that the vector of predicted states satisfies the following equality
\begin{equation}\label{eq:vectorized-states}
	\mbf{X}_k = \mcal{A} \mbf{x}_k + \mcal{B} \mbf{U}_k + \mcal{D}	
\end{equation}

where, using the matrices derived from the linear time-varying (LTV)
approximation of Section~\ref{sec:mpc}, we have:
\begin{equation*}
	\begin{aligned}
	\mcal{A} = \begin{bmatrix}
		\mbb{I}\\
		\mbf{A}_{d,k}\\
		\mbf{A}_{d,k+1} \mbf{A}_{d,k}\\
		\vdots\\
		\mbf{A}_{d,k+N-2} \ldots \mbf{A}_{d,k}
	\end{bmatrix}, 
	\quad
	\mcal{B} &= \begin{bmatrix}
		\mbf{0} & \mbf{0} & \ldots & \mbf{0} & \mbf{0}\\
		\mbf{B}_{d,k} & \mbf{0} & \ldots & \mbf{0} & \mbf{0}\\
		\mbf{A}_{d,k+1} \mbf{B}_{d,k} & \mbf{B}_{d,k+1} & \ldots & \mbf{0} & \mbf{0}\\
		\vdots & \vdots & \ddots & \vdots\\
		\mbf{A}_{d,k+N-2} \ldots \mbf{A}_{d,k} \mbf{B}_{d,k} & \ldots & \ldots &
		\mbf{B}_{d,k+N-2} & \mbf{0}
	\end{bmatrix},\\
		\text{and}\quad \mcal{D} &= \begin{bmatrix}
		\mbf{0}\\
		\mbf{d}_k\\
		\mbf{A}_{d,k+1} \mbf{d}_k + \mbf{d}_{k+1}\\
		\mbf{A}_{d,k+2} \mbf{A}_{d,k+1} \mbf{d}_k + \mbf{A}_{d,k+2}
		\mbf{d}_{k+1} + \mbf{d}_{k+2}\\
		\vdots
	\end{bmatrix}
	\end{aligned}
\end{equation*}

Consequently, by also introducing the large notation for the state $\mcal{Q} =
\text{diag}(\mbf{Q}, \ldots, \mbf{Q})$ and input $\mcal{R} =
\text{diag}(\mbf{R}, \ldots, \mbf{R})$ weighting matrices, it is possible
to rewrite the cost function $\mbf{J}$ of the
optimization problem~\ref{eq:mpc-problem} as a large matrix equation:
\begin{equation*}
	\begin{aligned}
		\mbf{J} &= \Delta \mbf{X}_k^T \mcal{Q} \Delta \mbf{X}_k + \Delta \mbf{U}_k^T
	\mcal{R} \Delta \mbf{U}_k\\
				 & = \left( \mbf{X}_k - \mbf{X}_{\text{ref},k} \right)^T
				 \mcal{Q} \left( \mbf{X}_k - \mbf{X}_{\text{ref},k} \right) +
				 \left( \mbf{U}_k - \mbf{U}_{\text{ref},k} \right)^T \mcal{R}
				 \left( \mbf{U}_k - \mbf{U}_{\text{ref},k} \right)\\
				 &= \mbf{U}_k^T \left(
						\mcal{B}^T \mcal{Q} \mcal{B} + \mcal{R}
					\right) \mbf{U}_k 
					+ 2 \left(
						\mbf{x}_k^T \mcal{A}^T \mcal{Q} \mcal{B} 
						+ \mcal{D}^T \mcal{Q} \mcal{B}
						- \bar{\mbf{X}}_k^T \mcal{Q} \mcal{B}
						- \bar{\mbf{U}}_k^T \mcal{R}
					\right) \mbf{U}_k\\
				 &= \mbf{U}_k^T \mbf{H}_k \mbf{U}_k + 2 \mbf{f}_k^T \mbf{U}_k
	\end{aligned}
\end{equation*}

which is a quadratic function of the input sequence $\mbf{U}_k$ and can hence be
efficiently solved with solvers such as \ttt{MATLAB}'s \ttt{quadprog} function.
Moreover, state and input constraints of the form $-\mbf{x}_{\min} \leq
\mbf{x}_k \leq \mbf{x}_{\max}$ and $-\mbf{u}_{\min} \leq \mbf{u}_k \leq
\mbf{u}_{\max}$ 
can be rewritten as
\begin{equation*}
	\boldsymbol{\varepsilon} \mbf{x}_k \leq \mbf{f}_{\mbf{x}} = \begin{bmatrix}
		\mbf{x}_{\max}\\
		\mbf{x}_{\min}
	\end{bmatrix}
	\quad \text{and} \quad
	\boldsymbol{\varepsilon} \mbf{u}_k \leq \mbf{f}_{\mbf{u}} = \begin{bmatrix}
		\mbf{u}_{\max}\\
		\mbf{u}_{\min}
	\end{bmatrix}
	\quad \text{with} \quad
	\boldsymbol{\varepsilon} = \begin{bmatrix}
		+1\\
		-1
	\end{bmatrix}
\end{equation*}

Hence, by stacking the constraints for all time steps as follows:
\begin{equation*}
	\mathcal{E}_{\mbf{x}} = \text{diag}(\boldsymbol{\varepsilon}, \ldots,
	\boldsymbol{\varepsilon}), 
	\quad
	\mcal{F}_{\mbf{x}} = \begin{bmatrix}
		\mbf{f}_{\mbf{x}}\\
		\vdots\\
		\mbf{f}_{\mbf{x}}
	\end{bmatrix},
	\quad
	\mathcal{E}_{\mbf{u}} = \text{diag}(\boldsymbol{\varepsilon}, \ldots,
	\boldsymbol{\varepsilon}), 
	\quad
	\mcal{F}_{\mbf{u}} = \begin{bmatrix}
		\mbf{f}_{\mbf{u}}\\
		\vdots\\
		\mbf{f}_{\mbf{u}}
	\end{bmatrix}
\end{equation*}

and using the equality~\ref{eq:vectorized-states}, the constraints can be
expressed as the compact inequality
\begin{equation*}
	\mcal{E} \mbf{U}_k \leq \mcal{F}
\end{equation*}

where
\begin{equation*}
	\mcal{E} = \begin{bmatrix}
		\mcal{E}_{\mbf{u}}\\
		\mcal{E}_{\mbf{x}} \mcal{B}
	\end{bmatrix}
	\quad \text{and} \quad
	\mcal{F} = \begin{bmatrix}
		\mcal{F}_{\mbf{u}}\\
		\mcal{F}_{\mbf{x}} - \mcal{E}_{\mbf{x}} (\mcal{A} \mbf{x}_k + \mcal{D})
	\end{bmatrix}
\end{equation*}

The final QP problem resulting from the \itt{dense} formulation can hence be
written as:
\begin{equation}
	\begin{aligned}
		\min_{\mbf{U}_k} \quad & \mbf{U}_k^T \mbf{H}_k \mbf{U}_k + 2 \mbf{f}_k^T
		\mbf{U}_k\\
		\text{s.t.} \quad & \mcal{E} \mbf{U}_k \leq \mcal{F}
	\end{aligned}
\end{equation}

\pagebreak
\subsection{Sparse Formulation}\label{app:sparse-formulation}

Given an MPC prediction horizon $N$, the \itt{sparse} formulation can be obtained by first
introducing the vectorized notation
\begin{equation*}
	\begin{aligned}
		\delta \mbf{X}_k &= \mbf{X}_k - \bar{\mbf{X}}_k = \begin{bmatrix}
			\delta \mbf{x}_k^T & \delta \mbf{x}_{k+1}^T & \ldots & \delta
			\mbf{x}_{k+N-1}^T
		\end{bmatrix}^T\\
		\delta \mbf{U}_k &= \mbf{U}_k - \bar{\mbf{U}}_k = \begin{bmatrix}
			\delta \mbf{u}_k^T & \delta \mbf{u}_{k+1}^T & \ldots & \delta
			\mbf{u}_{k+N-1}^T
		\end{bmatrix}^T
	\end{aligned}
\end{equation*}

so that the vector of state displacements from the operating points satisfies the
equality
\begin{equation}\label{eq:vectorized-states-displacement}
	\delta \mbf{X}_k = \mcal{A} \delta \mbf{X}_k + \mcal{B} \delta \mbf{U}_k + \mcal{D}	 \delta \mbf{x}_k
\end{equation}

where, using the matrices derived from the linear time-varying (LTV)
approximation of Section~\ref{sec:mpc}, we have:
{\fontsize{9}{12}\selectfont
\begin{equation*}
	% \begin{aligned}
	\mcal{A} = \begin{bmatrix}
		\mbf{0} & \mbf{0} & \ldots & \mbf{0} & \mbf{0}\\
		\mbf{A}_{d,k} & \mbf{0} & \ldots & \mbf{0} & \mbf{0}\\
		\mbf{0} & \mbf{A}_{d,k+1} & \ldots & \mbf{0} & \mbf{0}\\
		\vdots & \vdots & \ddots & \vdots\\
		\mbf{0} & \mbf{0} & \ldots & \mbf{A}_{d,k+N-1} & \mbf{0}
	\end{bmatrix}, 
	\mcal{B} = \begin{bmatrix}
		\mbf{0} & \mbf{0} & \ldots & \mbf{0} & \mbf{0}\\
		\mbf{B}_{d,k} & \mbf{0} & \ldots & \mbf{0} & \mbf{0}\\
		\mbf{0} & \mbf{B}_{d,k+1} & \ldots & \mbf{0} & \mbf{0}\\
		\vdots & \vdots & \ddots & \vdots\\
		\mbf{0} & \mbf{0} & \ldots & \mbf{B}_{d,k+N-1} & \mbf{0}
	\end{bmatrix},
	\mcal{D} = \begin{bmatrix}
		\mbb{I}\\
		\mbf{0}\\
		\vdots\\
		\mbf{0}\\
	\end{bmatrix}
	% \end{aligned}
\end{equation*}
}

The large notation matrices for the state $\mcal{Q} = \text{diag}(\mbf{Q},
\ldots, \mbf{Q})$ and input $\mcal{R} = \text{diag}(\mbf{R}, \ldots, \mbf{R})$
weighting matrices are also introduced and joint in the matrix $\mbf{V}$ as
follows:
\begin{equation*}
	\mbf{V} = \begin{bmatrix}
		\mcal{Q} & \mbf{0}\\
		\mbf{0} & \mcal{R}
	\end{bmatrix}
\end{equation*}

Consequently, the cost function $\mbf{J}$ of the optimization
problem~\ref{eq:mpc-problem} can be rewritten as a quadratic function of the
augmented state and input variables:
\begin{equation*}
	\mbf{Z}_k = \begin{bmatrix}
		\mbf{X}_k\\
		\mbf{U}_k
	\end{bmatrix}
	\quad \text{and} \quad
	\mbf{Z_{\text{ref},k}} = \begin{bmatrix}
		\mbf{X}_{\text{ref},k}\\
		\mbf{U}_{\text{ref},k}
	\end{bmatrix}
\end{equation*}

as
\begin{equation*}
	\begin{aligned}
		\mbf{J} &= \Delta \mbf{X}_k^T \mcal{Q} \Delta \mbf{X}_k + \Delta \mbf{U}_k^T
	\mcal{R} \Delta \mbf{U}_k\\
				&= \Delta \mbf{Z}_k^T \mbf{V} \Delta \mbf{Z}_k\\
				&= \left( \mbf{Z}_k - \mbf{Z}_{\text{ref},k} \right)^T \mbf{V}
				\left( \mbf{Z}_k - \mbf{Z}_{\text{ref},k} \right)\\
				&= \mbf{Z}_k^T \mbf{V} \mbf{Z}_k + 2 \left( -
				\mbf{Z}_{\text{ref},k}^T \mbf{V} \right) \mbf{Z}_k\\
				&= \mbf{Z}_k^T \mbf{H}_k \mbf{Z}_k + 2 \tilde{\mbf{f}}_k^T \mbf{Z}_k
	\end{aligned}
\end{equation*}

Equivalently to the \itt{dense} formulation, state and input constraints are
also expressed in the compact matrix inequality
\begin{equation*}
	\mcal{E}_{\mbf{Z}} \mbf{Z}_k \leq \mcal{F}_{\mbf{Z}}
\end{equation*}

where
\begin{equation*}
	\mcal{E}_{\mbf{Z}} = \begin{bmatrix}
		\mcal{E}_{\mbf{x}} & \mbf{0}\\
		\mbf{0} & \mcal{E}_{\mbf{u}}
	\end{bmatrix}
	\quad \text{and} \quad
	\mcal{F}_{\mbf{Z}} = \begin{bmatrix}
		\mcal{F}_{\mbf{x}}\\
		\mcal{F}_{\mbf{u}}
	\end{bmatrix}
\end{equation*}

and $\mcal{E}_{\mbf{x}}$, $\mcal{E}_{\mbf{u}}$, $\mcal{F}_{\mbf{x}}$ and
$\mcal{F}_{\mbf{u}}$
assume the same meaning as in Appendix~\ref{app:dense-formulation}.
Additionally, equation~\ref{eq:vectorized-states-displacement} implies the
following equality constraints on the augmented state and input variables:
\begin{equation*}
	(~\ref{eq:vectorized-states-displacement}~) 
	\quad \Longrightarrow \quad
	\begin{bmatrix}
		\mbb{I} - \mcal{A} & -\mcal{B}
	\end{bmatrix}
	\delta \mbf{Z}_k = \mcal{D} \delta \mbf{x}_k
	\quad \Longrightarrow \quad
	\mcal{E}_{\text{eq.}} \mbf{Z}_k = \mcal{F}_{\text{eq.}}
\end{equation*}

where
\begin{equation*}
	\mcal{E}_{\text{eq.}} = \begin{bmatrix}
		\mbb{I} - \mcal{A} & -\mcal{B}
	\end{bmatrix}
	\quad \text{and} \quad
	\mcal{F}_{\text{eq.}} = \mcal{D} \delta \mbf{x}_k + \begin{bmatrix}
		\mbb{I} - \mcal{A} & -\mcal{B}
\end{bmatrix} \bar{\mbf{Z}}_k
\end{equation*}

The final QP problem resulting from the \itt{sparse} formulation can hence be
written as:
\begin{equation}
	\begin{aligned}
		\min_{\mbf{Z}_k} \quad & \mbf{Z}_k^T \mbf{V} \mbf{Z}_k + 2
		\tilde{\mbf{f}}_k^T
		\mbf{Z}_k\\
		\text{s.t.} \quad & \mcal{E}_{\mbf{Z}} \mbf{Z}_k \leq \mcal{F}_{\mbf{Z}}\\
		& \mcal{E}_{\text{eq.}} \mbf{Z}_k = \mcal{F}_{\text{eq.}}
	\end{aligned}
\end{equation}
































% \subsection{Notice on Generative Tools Usage}\label{app:generative-tools}

% Generative AI tools have been used as a support for the development of this
% project. In particular, the
% \href{https://en.wikipedia.org/wiki/Microsoft_Copilot}{Copilot \faLink} generative tool
% based on \href{https://en.wikipedia.org/wiki/GPT-4}{OpenAI GPT 4o \faLink} model has
% been used as assistance medium in performing the following tasks:

% \begin{itemize}
% 	 \item writing documentation and comments in implemented functions by
% 		 adhering to the
% 		 \href{https://numpydoc.readthedocs.io/en/latest/format.html}{NumPy
% 		 Style Guide \faLink}

% 	\item improving variable naming and code readability 
	
% 	\item minor bug fixing in implemented functions

% 	\item grammar and spelling check both in the repository
% 		\ttt{README}~\cite{github} and in this report

% 	\item tweaking aesthetic improvements in the plots of this report

% 	\item formatting table of results in this report
% \end{itemize}

\end{document}


