\documentclass[../main.tex]{subfiles}
\graphicspath{{\subfix{../images/}}} % Images path

\begin{document}

\section{Extended Kalman Filter (EKF)}\label{sec:ekf}

The previously introduced MPC algorithm relies on the knowledge of the complete
state of the system at any given time. However, in many practical applications,
the complete state of a system cannot always be fully measured. Moreover,
disturbances and noise can affect the measurements, leading to inaccurate
predictions and control actions.\\
To address these issues for the studied non-linear systems, the Extended Kalman
Filter (EKF) has been implemented. The EKF is the non-linear version of the
Kalman Filter, which implies linearization around an estimate of the current
mean and covariance.\\
To explain the implementation of the EKF in this study, it's assumed that the
equivalent discrete-time versions of the state equation and the output
transformation of the implemented non-linear systems are affected by noise as
follows:
\begin{equation}
  \begin{aligned}
	  \dot{\mbf{x}}_k &= \mbf{f}(\mbf{x}_{k-1}, \mbf{u}_{k-1}) + \mbf{w}_k \\
	  \mbf{y}_k &= \mbf{g}(\mbf{x}_k) + \mbf{v}_k
  \end{aligned}
\end{equation}

Where $\mbf{w}_k$ and $\mbf{v}_k$ are process and observation noises, which
are both assumed to be zero-mean multivariate Gaussian noises with covariance
matrices $\tilde{\mbf{Q}}_k$ and $\tilde{\mbf{R}}_k$, respectively.\\
The function $\mbf{g}(\cdot)$ is instead the output transformation function
which can be used to compute the predicted measurements $\mbf{y}_k$ from the estimated
states. In general, the measurements don't correspond to the full state of the
system, which has to be estimated by introducing a gain matrix $\mbf{K}_k$ that
is computed at each time step.
In the experiments conducted for this study, a natural choice for the
implemented models has been to simulate a measurement of the flat outputs only,
since the reconstruction of the full state of the system is then guaranteed by the
previously introduced \itt{differential flatness} property.\\
To derive the Kalman gain $\mbf{K}_k$, $\mbf{f}$ and $\mbf{g}$
cannot be applied to the covariance directly. Instead, a matrix of partial
derivatives (the \itt{Jacobian}) is computed and evaluated at the current
estimate of the state performing in this way a linearization. Such matrices are then used in the
classic Kalman Filter equations to compute the desired gain.\\
The whole process can be divided in two main phases: the \itt{prediction}
(eq.~\ref{eq:ekf-prediction}), where a first estimate of the state is computed
as well as its covariance, and the \itt{update} (eq.~\ref{eq:ekf-update}),
where the information from the measurements is used to correct and refine the state
estimate and its covariance.

\begin{itemize}
	\item \bft{Prediction}:
	\begin{equation}\label{eq:ekf-prediction}
		\begin{aligned}
			\hat{\mbf{x}}_{k | k-1} = \mbf{f}(\hat{\mbf{x}}_{k-1 | k-1},
			\mbf{u}_{k-1}) \quad & \quad \text{Predicted state estimate} \\
			\mbf{P}_{k | k-1} = \mbf{A}_k \mbf{P}_{k-1 | k-1} \mbf{A}_k^T +
			\tilde{\mbf{Q}}_{k - 1} \quad & \quad \text{Predicted covariance estimate}
		\end{aligned}
	\end{equation}

\item \bft{Update}:
	\begin{equation}\label{eq:ekf-update}
		\begin{aligned}
			\mbf{K}_k = \mbf{P}_{k | k-1} \mbf{C}_k^T \left(
				\mbf{C}_k \mbf{P}_{k | k-1} \mbf{C}_k^T + \tilde{\mbf{R}}_k
			\right)^{-1} \quad & \quad \text{Kalman gain} \\
			\hat{\mbf{x}}_{k | k} = \hat{\mbf{x}}_{k | k-1} + \mbf{K}_k \left(
								\mbf{y}_k - \mbf{g}(\hat{\mbf{x}}_{k | k-1})
				\right) \quad & \quad \text{Updated state estimate} \\
				\mbf{P}_{k | k} = \left( \mbb{I} - \mbf{K}_k \mbf{C}_k \right)
				\mbf{P}_{k | k-1} \quad & \quad \text{Updated covariance
				estimate}
		\end{aligned}
	\end{equation}
\end{itemize}

where $\mbf{A}_k$ and $\mbf{C}_k$ are the Jacobians of $\mbf{f}$ and $\mbf{g}$
respecively computed as:
\begin{equation*}
		\mbf{A}_k = \left. \frac{\partial \mbf{f}}{\partial \mbf{x}}
			\right|_{\hat{\mbf{x}}_{k-1 | k-1}, \mbf{u}_{k}} \quad\quad
			\text{and} \quad\quad
		\mbf{C}_k = \left. \frac{\partial \mbf{g}}{\partial \mbf{x}} \right|_{\hat{\mbf{x}}_{k | k-1}}
\end{equation*}

The state estimate obtained from the EKF process can also be used internally by
the MPC algorithm. In particular, at each iteration of
the MPC algorithm, the initial state for the given prediction horizon can be
estimated by the EKF from the measurements of the flat outputs only, while future states of
the horizon are predicted using the non-linear dynamics and the previously
explained MPC process.
This strategy allows for the inclusion of both process and measurement noise
only in the first step of each MPC iteration and doesn't propagate such
uncertainty throughout the remaining prediction horizon. Nevertheless, in the
presence of noise, this approach resulted in far better input sequences for
reference tracking than
the ones obtained by ignoring the disturbances in the MPC algorithm.
% This strategy allows for the account of both process and measurement
% noise in the MPC algorithm, even if such uncertainty is then not propagated
% throughout the prediction horizon. At least, this allows to obtain better input
% sequences than the ones that would be obtained by ignoring noise and
% disturbances.\\

\end{document}

