\documentclass[../main.tex]{subfiles}
\graphicspath{{\subfix{../images/}}} % Images path

\begin{document}

\section{MPC Problem Formulation}\label{sec:mpc}

The Model Predictive Control (MPC) algorithm is based on the solution of an
optimization problem over a finite horizon of length $N$. In general, the formulation of such
optimization problem requires a discrete-time representation of the model
dynamics. Moreover, in order to cast the MPC problem as a Quadratic Program
(QP) one and hence solve it efficiently, the model must also be linear.
The models introduced in Section~\ref{sec:models} are not only defined in
continuous-time, but also highly non-linear. To overcome this issues, a Linear
Time Varying (LTV) approximation of the dynamics is here presented. \mbox{Then, two
formulations of the MPC algorithm as a QP problem are introduced.}

\subsection{Linear Time Varying (LTV) Model}

The LTV model is obtained by successive linearization and discretization of the
non-linear dynamics $\dot{\mbf{x}} = \mbf{f}(\mbf{x}, \mbf{u})$ around a set of nominal state and input trajectories sampled
at a fixed regular time intervals $T_s$:
\begin{equation*}
	\begin{aligned}
		\bar{\mbf{x}}_k &= \bar{\mbf{x}}(k T_s), \quad k = k_0, \ldots, k_0 + N + 1\\
		\bar{\mbf{u}}_k &= \bar{\mbf{u}}(k T_s), \quad k = k_0, \ldots, k_0 + N
	\end{aligned}
\end{equation*}

Hence, the system is first linearized around the nominal states and inputs as
follows:
\begin{equation*}
	\delta \mbf{x}(t) = \mbf{A}_k \delta \mbf{x}(t) + \mbf{B}_k \delta
	\mbf{u}(t)
\end{equation*}
\begin{equation}\label{eq:linearization}
	\mbf{A}_k = \left. \frac{\partial \mbf{f}}{\partial \mbf{x}}
		\right|_{\bar{\mbf{x}}_k, \bar{\mbf{u}}_k}, \quad
		\mbf{B}_k = \left. \frac{\partial \mbf{f}}{\partial \mbf{u}}
			\right|_{\bar{\mbf{x}}_k, \bar{\mbf{u}}_k}
\end{equation}
\begin{equation*}
	\delta \mbf{x}(t) = \mbf{x}(t) - \bar{\mbf{x}}_k, \quad
	\delta \mbf{u}(t) = \mbf{u}(t) - \bar{\mbf{u}}_k
\end{equation*}

with $k T_s \leq t < (k + 1) T_s$ for $k = k_0, \ldots, k_0 + N$. Then, the obtained
linear system is discretized using first order forward Euler method:
\begin{equation*}
	\delta \mbf{x}_{k+1} = \mbf{A}_{d,k} \delta \mbf{x}_k + \mbf{B}_{d,k} \delta
	\mbf{u}_k
\end{equation*}
\begin{equation}\label{eq:discretization}
	\mbf{A}_{d,k} = \mbb{I} + T_s \mbf{A}_k, \quad \mbf{B}_{d,k} = T_s
	\mbf{B}_k
\end{equation}
\begin{equation*}
	\delta \mbf{x}_k = \mbf{x}_k - \bar{\mbf{x}}_k, \quad
	\delta \mbf{u}_k = \mbf{u}_k - \bar{\mbf{u}}_k
\end{equation*}

This can eventually be rewritten as:
\begin{equation}
	\begin{aligned}
		\mbf{x}_{k+1} &= \mbf{A}_{d,k} \mbf{x}_k + \mbf{B}_{d,k} \mbf{u}_k +
		\mbf{d}_k\\
		\mbf{d}_k &= \bar{\mbf{x}}_{k+1} - \mbf{A}_{d,k} \bar{\mbf{x}}_k -
		\mbf{B}_{d,k} \bar{\mbf{u}}_k
	\end{aligned}
\end{equation}

Using this LTV model, the MPC controller is able to solve an optimization
problem for each $k^{th}$ time step in the prediction horizon $N$. In particular, the
result of the optimization will be the optimal input sequence that minimizes the
cost function in the prediction horizon, from which only the first input will
actually be applied. Hence, in successive applications of the algorithm, a
nominal input sequence for future linearizations can be obtained as the
remaining part of the optimal input sequence\footnote{With the last input
duplicated as $N$ nominal inputs are needed.} while the nominal states can be
obtained by applying the nominal inputs to the non-linear dynamics. For the
first iteration of the MPC algorithm, the reference input sequence can be used
even if it will result in an inaccurate linearization.

\subsection{MPC Formulations}

The Model Predictive Control problem with the LTV
model approximation of the non-linear dynamics is formulated as the optimization
problem~\ref{eq:mpc-problem}, which is solved at every time instance $t$.
\begin{equation}\label{eq:mpc-problem}
	\begin{aligned}
		\min_{\mbf{U}_{t \to t+N | t}} \quad & \sum_{k=t}^{t+N} \left( \Delta
			\mbf{x}_k^T \mbf{Q} \Delta \mbf{x}_k + \Delta \mbf{u}_k^T \mbf{R}
			\Delta \mbf{u}_k \right)\\
		\text{s.t.} \quad & \mbf{x}_{k+1 | t} = \mbf{A}_{d,k | t} \mbf{x}_{k |
		t} + \mbf{B}_{d,k | t} \mbf{u}_{k | t} + \mbf{d}_{k | t}\\
						  & \mbf{d}_{k | t} = \bar{\mbf{x}}_{k+1 | t} -
						  \mbf{A}_{d,k | t} \bar{\mbf{x}}_{k | t} - \mbf{B}_{d,k
						  | t} \bar{\mbf{u}}_{k | t}\\
						  & \mbf{x}_{k | t} \in \mcal{X}, \quad k = t, \ldots,
						  t+N\\
						  & \mbf{u}_{k | t}
						  \in \mcal{U}, \quad k = t, \ldots, t+N+1\\
		\end{aligned}
\end{equation}

Where $\mbf{U}_{t \to t+N | t} = \begin{bmatrix} \mbf{u}_{t | t} & \ldots &
\mbf{u}_{t+N | t} \end{bmatrix}^T$ is the input sequence to be optimized, while
$\Delta \mbf{x}_k = \mbf{x}_{k | t} - \bar{\mbf{x}}_{k}$ and $\Delta \mbf{u}_k =
\mbf{u}_{k | t} - \bar{\mbf{u}}_{k}$ are the differences between the predicted
states and the reference trajectory. The matrices $\mbf{Q}$ and $\mbf{R}$ are
the weighting matrices for the state and input errors, respectively.
This general expression of the MPC problem can be further casted into a QP
problems in two different formulations: the \itt{dense} formulation and the
\itt{sparse} formulation.


\end{document}

