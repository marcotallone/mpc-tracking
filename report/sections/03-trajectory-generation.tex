\documentclass[../main.tex]{subfiles}
\graphicspath{{\subfix{../images/}}} % Images path

\begin{document}

\section{Trajectory Generation}\label{sec:trajectory-generation}

Prior to the implementation of the MPC algorithm for reference tracking, it's
compulsory to properly define the reference trajectories that the system should
follow. In particular, any desired trajectory should be feasible for the given system.
Formally, given a continuous-time reference trajectory 
$\begin{bmatrix} \mbf{x}_{\text{ref}}(t) & \mbf{u}_{\text{ref}}(t) \end{bmatrix}^T$, in order
for it to be feasible, it must satisfy the differential equations of the system
dynamics within the given constraints:
\begin{equation}
	\begin{aligned}
		&\dot{\mbf{x}}_{\text{ref}}(t) = \mbf{f}(\mbf{x}_{\text{ref}}(t), \mbf{u}_{\text{ref}}(t))\\
		&\mbf{x}_{\text{ref}} \in \mcal{X}, \quad \mbf{u}_{\text{ref}} \in \mcal{U}
	\end{aligned}
\end{equation}

where $\mcal{X}$ and $\mcal{U}$ are the sets of feasible states and inputs for
the system while the function $\mbf{f}(\cdot, \cdot): \mbb{R}^n \times \mbb{R}^m \rightarrow
\mbb{R}^n$ describes the non-linear state dynamics.\\
Th approaches considered in this study for the feasible trajectory generation task rely on the definition of \itt{differential flatness}
introduced by \itt{Murray}~\cite{murray}. Formally, a non-linear
system is \itt{differentially flat} if there exists a function $\Gamma(\cdot)$
such that
\begin{equation*}
	\mbf{z} = \Gamma(\mbf{x}, \mbf{u}, \dot{\mbf{u}}, \ldots, \mbf{u}^{(p)})
\end{equation*}

where $\mbf{z}$ are the so-called \itt{flat outputs}. For a differentially flat
system, feasible trajectories can be written as functions of the flat outputs
$\mbf{z}$ and their derivatives. Hence, the general idea is to find a mapping 
\begin{equation*}
	\begin{aligned}
		\mbf{x} &= \mbf{x}(\mbf{z}, \dot{\mbf{z}}, \ldots, \mbf{z}^{(q)})\\
		\mbf{u} &= \mbf{u}(\mbf{z}, \dot{\mbf{z}}, \ldots, \mbf{z}^{(q)})
	\end{aligned}
\end{equation*}

between the original states and the flat outputs such that all states and inputs
can be determined from these outputs without integration. Then, a feasible
trajectory can simply  be obtained by only providing the reference values for the flat
outputs (and possibly their derivatives), since the remaining reference states
and associated inputs can be recovered from the mapping.\\
It's easy to show that, for the unicycle model described by
equation~\eqref{eq:unicycle-dynamics}, the Cartesian coordinates $z_1 = x$ and $z_2 = y$ are flat outputs. The
remaining heading angle state\footnote{In the implemented models, the
	$atan2(\cdot, \cdot)$ function has been used instead of the normal arctangent to
obtain the correct heading angle in the range $[-\pi, \pi]$.} and the control inputs can in fact be obtained
from these two outputs and their first order derivatives as shown in
equations~\eqref{eq:unicycle-flat-outputs}.
\begin{equation}\label{eq:unicycle-flat-outputs}
	\theta = \arctan\left(\frac{\dot{z_2}}{\dot{z_1}}\right), \quad
	\omega_1 = \frac{2 \sqrt{\dot{z_1}^2 + \dot{z_2}^2} + L \dot{\theta}}{2 r},
	\quad
	\omega_2 = \frac{2 \sqrt{\dot{z_1}^2 + \dot{z_2}^2} - L \dot{\theta}}{2 r}
\end{equation}

For what concerns the helicopter model instead, a set of flat outputs is given
by the state components
\begin{equation*}
	z_1 = x_I, \quad z_2 = y_I, \quad z_3 = z_I, \quad z_4 = \psi
\end{equation*}

which allow to obtain the remaining states and inputs according to
equations~\eqref{eq:helicopter-flat-outputs}.

\begin{equation}\label{eq:helicopter-flat-outputs}
	\begin{aligned}
		\dot{x}_B &= \cos(z_4) \dot{z}_1 + \sin(z_4) \dot{z}_2\\
		\dot{y}_B &= -\sin(z_4) \dot{z}_1 + \cos(z_4) \dot{z}_2\\
		\dot{z}_B &= \dot{z}_3\\
		\dot{\psi} &= \dot{z}_4\\
		u_x &= \frac{1}{b_x} \left( cos(z_4) \left[ \ddot{z}_1 - k_x \dot{z}_1
		\right] + sin(z_4) \left[ \ddot{z}_2 - k_x \dot{z}_2 \right] \right)\\
		u_y &= \frac{1}{b_y} \left( cos(z_4) \left[ \ddot{z}_2 - k_y \dot{z}_2 \right] + sin(z_4) \left[ -\ddot{z}_1 + k_y \dot{z}_1 \right] \right)\\
		u_z &= \frac{1}{b_z} \left( \ddot{z}_3 + g \right)\\
		u_{\psi} &= \frac{1}{b_{\psi}} \left( \ddot{z}_4 - k_{\psi} \dot{z}_4
			\right)
	\end{aligned}
\end{equation}

After having identified the flat outputs and the required mappings to
reconstruct state and inputs components, the
generation of the feasible trajectory can then be conducted in two ways.\\

\pagebreak
The first approach, which is the one used in this study, consists in providing
an analytical expression $\mbf{z}(t)$ for the evolution of the flat outputs over time and
then sampling  the resulting trajectory at discrete time steps
$\{t_k\}_{k=1}^{N_{guide}}$	to obtain a finite set of $N_{guide}$ reference
points $\{(\mbf{x}_{\text{ref}, k}, \mbf{u}_{\text{ref}, k})\}_{k=1}^{N_{guide}}$.
In particular, two trajectories have been generated with this technique:
\begin{itemize}
	
	\item a \bft{circular} trajectory of radius $r_0$ whose flat
		outputs are defined respectively by the parametric equations:
		\vspace{-0.5cm}
		\begin{multicols}{2}
			\begin{equation*}
				\begin{cases}
					z_1(t) = r_0 \cos(t)\\
					z_2(t) = r_0 \sin(t)\\
				\end{cases}
			\end{equation*}
			\begin{center}
				for the unicycle model\\
			\end{center}
		\columnbreak
			\begin{equation*}
				\begin{cases}
					z_1(t) = r_0 \cos(t)\\
					z_2(t) = r_0 \sin(t)\\
					z_3(t) = 0\\
					z_4(t) = \text{atan2}\left(\dot{z}_2(t), \dot{z}_1(t)\right)
				\end{cases}
			\end{equation*}
			\begin{center}
				for the helicopter model
			\end{center}
		\end{multicols}

	\item a \bft{leminscate} (or ``\itt{figure-eight}'') trajectory with
		parameter $a$ described by:
		\vspace{-0.1cm}
		\begin{multicols}{2}
			\begin{equation*}
				\begin{cases}
					z_1(t) = \frac{a \sqrt{2} \cos(t)}{\sin(t)^2 + 1}\\
					z_2(t) = \frac{a \sqrt{2} \cos(t) \sin(t)}{\sin(t)^2 + 1}
				\end{cases}
			\end{equation*}
			\begin{center}
				for the unicycle model\\
			\end{center}
		\columnbreak
			\begin{equation*}
				\begin{cases}
					z_1(t) = \frac{a \sqrt{2} \cos(t)}{\sin(t)^2 + 1}\\
					z_2(t) = \frac{a \sqrt{2} \cos(t) \sin(t)}{\sin(t)^2 + 1}\\
					z_3(t) = 0\\
					z_4(t) = \text{atan2}\left(\dot{z}_2(t), \dot{z}_1(t)\right)
				\end{cases}
			\end{equation*}
			\begin{center}
				for the helicopter model
			\end{center}
		\end{multicols}

\end{itemize}

In the conducted experiments, the models performed a full
revolution\footnote{However, the implementation easily allows to perform more than one
lap around the given shapes.} around the respective trajectories in a determined amount of
time.
Specifically, due to the discretization with sampling time $T_s$ later introduced in Section~\ref{sec:mpc}, the total time to complete the
trajectory has been set to
$T_{end} = N_{guide} \cdot T_s$ for simplicity. In this way, at each iteration,
the discretized system has one time-step to reach the next reference point.
Moreover, notice that the discretization assumes a constant input applied to
the system at each time-step. Hence, a possibility would be to just sample the
reference inputs at the beginning of each time interval $\mbf{u}_{\text{ref}, k} =
\mbf{u}_{\text{ref}}(t_k)$. However, due to the continuous-time nature of the 
inputs obtained with this approach, a more realistic approximation of the
system's behavior can be obtained by averaging the values of the inputs over
each time interval:
\begin{equation*}
	\mbf{u}_{\text{ref}, k} = \frac{1}{N_{samples}} \sum_{i=1}^{N_{samples}}
	\mbf{u}_{\text{ref}}\left(t_k + \frac{i
	\cdot T_s}{N_{samples}} \right)
\end{equation*}

where $N_{samples}$ is the number of sub-samples taken in each time interval to
compute the average, which depends on the specific trajectory and model considered.\\
When an analytical expression for the flat outputs is not available, the
approach proposed by \itt{Murray}~\cite{murray} can be used to generate feasible
trajectories. In this case the values of the flat outputs (and possibly their
derivatives up to a determined order) for a minimal\footnote{At least $2$
points.} set of representative points of the desired trajectory are needed.
Then, between any two given points, the flat outputs can be parametrized as a
linear combination of a finite set of basis functions. The coefficients of this
linear combination are determined by simply solving a linear system of equations
involving the known values of the flat outputs and their derivatives at the
endpoints of the interval. This results in a parametrized analytical expression
for the flat outputs between the given points, which can then be resampled to
obtain a higher number of reference points.\\
Therefore, this approach presents the huge advantage of being able to generate
feasible and smooth point-to-point trajectories for any arbitrary set of initial
reference points.\\
In practice, however, to generate realistic inputs for most systems, high order
derivatives of the basis functions involved in the parametrization are often required,
especially when the desired trajectories present a high number of sharp turns
or sudden changes in direction. This turns out to be unpractical since the known
values for the flat outputs derivatives of the same order must be provided at
least for the initial set of points. Hence, although implemented, this method
has not been used in this study, but further information can be found in
Appendix~\ref{app:murray} or in the original work by
\itt{Murray}~\cite{murray}.

\end{document}
